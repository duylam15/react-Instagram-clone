Để hệ thống hoạt động đầy đủ cả frontend và backend, thực hiện theo các bước sau:

\begin{itemize}
    \item \textbf{Bước 1: Tải source code từ GitHub}
    \begin{itemize}
        \item Backend: \texttt{https://github.com/LocNguyenSGU/SocialMedia}
        \item Frontend: \texttt{https://github.com/duylam15/react-Instagram-clone}
    \end{itemize}
    
    \item \textbf{Bước 2: Tạo cơ sở dữ liệu MySQL}
    \begin{itemize}
        \item Tạo database với tên \texttt{socialMedia} trong MySQL.
        \item Đảm bảo các thông tin cấu hình như username/password được khai báo đúng trong file \texttt{application.properties} của backend.
    \end{itemize}
    
    \item \textbf{Bước 3: Chạy Redis}
    \begin{itemize}
        \item Dùng Docker: \texttt{docker run -d -p 6379:6379 redis}
    \end{itemize}
    
    \item \textbf{Bước 4: Chạy RabbitMQ}
    \begin{itemize}
        \item Dùng Docker: \\
        \texttt{docker run -d --hostname rabbit --name rabbitmq -p 5672:5672 -p 15672:15672 rabbitmq:3-management}
        \item Truy cập giao diện RabbitMQ tại \texttt{http://localhost:15672} \\
        Tài khoản mặc định: \texttt{guest / guest}
    \end{itemize}
    
    \item \textbf{Bước 5: Chạy backend (Spring Boot)}
    \begin{itemize}
        \item Mở project bằng IntelliJ IDEA hoặc IDE tương tự.
        \item Đảm bảo đã cài Java 21 trở lên.
        \item Chạy file \texttt{SocialMediaApplication.java} để khởi động hệ thống backend.
    \end{itemize}
    
    \item \textbf{Bước 6: Cài đặt và chạy frontend (ReactJS)}
    \begin{itemize}
        \item Mở thư mục frontend bằng VSCode.
        \item Cài thư viện bằng lệnh: \texttt{npm install}
        \item Sau khi cài xong, chạy project bằng: \texttt{npm run dev}
    \end{itemize}
\end{itemize}